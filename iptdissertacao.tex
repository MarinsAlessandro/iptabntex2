\documentclass[
	% -- opções da classe memoir --
	12pt,				% tamanho da fonte
	openright,			% capítulos começam em pág ímpar (insere página vazia caso preciso)
	twoside,			% ou oneside
%	oneside,			% ou twoside
	a4paper,			% tamanho do papel.
	% -- opções da classe abntex2 --
	chapter=TITLE,		% títulos de capítulos convertidos em letras maiúsculas
	%section=TITLE,		% títulos de seções convertidos em letras maiúsculas
	%subsection=TITLE,	% títulos de subseções convertidos em letras maiúsculas
	%subsubsection=TITLE,% títulos de subsubseções convertidos em letras maiúsculas
	%Sumario
	sumario=abnt-6027-2012,
	% -- opções do pacote babel --
	english,			% idioma adicional para hifenização
	brazil,				% o último idioma é o principal do documento
	]{abntex2}

% ---
% Pacotes fundamentais
% ---
\usepackage[T1]{fontenc}		% Selecao de codigos de fonte.
\usepackage{lastpage}			% Usado pela Ficha catalográfica
\usepackage{indentfirst}		% Indenta o primeiro parágrafo de cada seção.
\usepackage{color}				% Controle das cores
\usepackage{graphicx}			% Inclusão de gráficos
\usepackage[brazil]{babel}		% Para definir a língua - abntex2.cls já carrega
% Fonte Arial >
\usepackage{fontspec}
\usepackage{xltxtra}
\setromanfont[Mapping=tex-text]{Arial}
\setsansfont[Mapping=tex-text]{Arial}
% Fonte Arial <
\usepackage{setspace}			% Permite Configuracao de Espaçamento entre Linhas
\usepackage{microtype}			% Ajuda a ajustar hifenização
%\hyphenpenalty = 10000			% Remove a Hifenização
%\hyphenpenalty = 1000			% Reduz a Hifenização
\usepackage{type1cm}			% Fontes Escaláveis
\usepackage{float}				% Permite Posicionar as Figuras de Forma Correta
\usepackage{csquotes}			% Configuração adicional para aspas
\usepackage{lscape}				% Possibilita a criação de paginas em formato paisagem
\usepackage{enumerate}          % permite listas numeradas

\usepackage{iptabntex2}		% Pacote com Customizações do abntex2 + IPT para esta dissertaçao

\usepackage[alf,abnt-emphasize=bf,abnt-etal-list=2,abnt-etal-cite=2,abnt-etal-text=2,abnt-repeated-title-omit=yes]{abntex2cite}	% Citações padrão ABNT conforme orientação do IPT

% ---
% Informações de dados para CAPA e FOLHA DE ROSTO
% ---
\titulo{Título da Dissertação}
\autor{Autor}
\data{Ano}
\mes{Mês}
\local{Cidade}
\instituicao{\mbox{Instituto} de Pesquisas Tecnológicas do \mbox{Estado} de São Paulo}
\tipotrabalho{Dissertação de Mestrado}
\curso{Engenharia da Computação: Redes de Computadores}
\concentracao{Área de Concentração: Redes de \mbox{Computadores}}
\orientador{Prof. Dr. Orientador}
\orientadorafiliacao{Mestrado em Engenharia de Computação}
\membroum{Prof. Dr. Membro Um}
\membroumAfiliacao{Mestrado em Engenharia de Computação}
\membrodois{Prof. Dr. Membro Dois}
\membrodoisAfiliacao{Mestrado em Engenharia de Computação}
\palavraschave{Mestrado, IPT, LATEX, ABNTEX2.}
\keyword{Master, IPT, LATEX, ABNTEX2.}
%% Preambulo para Qualificacao
\preambulo{Exame de Qualificação apresentado ao \imprimirinstituicao~- IPT, como parte dos requisitos para a obtenção do título de Mestre em \imprimircurso}
% Preambulo para Defesa
%\preambulo{Dissertação de Mestrado apresentada ao \imprimirinstituicao~- IPT, como parte dos requisitos para a obtenção do título de Mestre em  \imprimircurso}

% informações do PDF
\makeatletter
\hypersetup{
plainpages=false,
colorlinks=true,
citecolor=black,
linkcolor=black,
urlcolor=black,
filecolor=black,
bookmarksopen=true,
pdftitle={\imprimirtitulo},
pdfauthor={\imprimirautor},
pdfsubject={\imprimirpreambulo},
pdfcreator={XeLaTeX com abnTeX2},
pdfkeywords={\imprimirpalavraschave},
bookmarksdepth=4
}
\makeatother
% ---

% ---
% compila o indice
% ---
%\makeindex
% ---


\usepackage{lipsum} 

% Início do documento
% ----
\begin{document}
\chapterstyle{iptabntex2}			% Estilo de capitulo adotado pelo IPT
%\sloppy							% Ajusta linhas sem a hifenização
\frenchspacing					% Retira espaço extra, obsoleto, entre as frases.

% ----------------------------------------------------------
% ELEMENTOS PRÉ-TEXTUAIS
% ----------------------------------------------------------
\pretextual
\imprimircapa

\imprimirfolhadeaprovacao

\imprimirfolhaderosto

% ---
% Inserir a ficha bibliografica
% ---

% Isto é um exemplo de Ficha Catalográfica, ou ``Dados internacionais de
% catalogação-na-publicação''. Você pode utilizar este modelo como referência.
% Porém, provavelmente a biblioteca da sua universidade lhe fornecerá um PDF
% com a ficha catalográfica definitiva após a defesa do trabalho. Quando estiver
% com o documento, salve-o como PDF no diretório do seu projeto e substitua todo
% o conteúdo de implementação deste arquivo pelo comando abaixo:
%
% \begin{fichacatalografica}
%     \includepdf{fig_ficha_catalografica.pdf}
% \end{fichacatalografica}
%\begin{fichacatalografica}
%	\vspace*{\fill}					% Posição vertical
%	\hrule							% Linha horizontal
%	\begin{center}					% Minipage Centralizado
%	\begin{minipage}[c]{12.5cm}		% Largura
%
%	\imprimirautor
%
%	\hspace{0.5cm} \imprimirtitulo  / \imprimirautor. --
%	\imprimirlocal, \imprimirdata-
%
%	\hspace{0.5cm} \pageref{LastPage} p. : il. (algumas color.) ; 30 cm.\\
%
%	\hspace{0.5cm} \imprimirorientadorRotulo~\imprimirorientador\\
%
%	\hspace{0.5cm}
%	\parbox[t]{\textwidth}{\imprimirtipotrabalho~--~\imprimirinstituicao,
%	\imprimirdata.}\\
%
%	\hspace{0.5cm}
%		1. Palavra-chave1.
%		2. Palavra-chave2.
%		I. Orientador.
%		II. Universidade xxx.
%		III. Faculdade de xxx.
%		IV. Título\\
%
%	\hspace{8.75cm} CDU 02:141:005.7\\
%
%	\end{minipage}
%	\end{center}
%	\hrule
%\end{fichacatalografica}
% ---



% ---
% Dedicatória
% ---
\begin{dedicatoria}
\vspace*{\fill}
\OnehalfSpacing
%\centering
%\noindent
Dedico este trabalho...

\lipsum[1]

\vspace*{\fill}
\end{dedicatoria}
% ---

% ---
% Agradecimentos
% ---
\begin{agradecimentos}
\vspace*{\fill}
\OnehalfSpacing
Gostaria de agradecer...

\lipsum[2]

\vspace*{\fill}
\end{agradecimentos}
% ---

% ---
% Epígrafe
% ---
%\begin{epigrafe}
%    \vspace*{\fill}
%	\begin{flushright}
%		\textit{``Não vos amoldeis às estruturas deste mundo, \\
%		mas transformai-vos pela renovação da mente, \\
%		a fim de distinguir qual é a vontade de Deus: \\
%		o que é bom, o que Lhe é agradável, o que é perfeito.\\
%		(Bíblia Sagrada, Romanos 12, 2)}
%	\end{flushright}
%\end{epigrafe}
% ---


\begin{resumo}
\normalsize

Resumo da dissertação em português

\lipsum[3]

\vspace{\onelineskip}

\noindent
\textbf{Palavras-chave:} \imprimirpalavraschave
\end{resumo}

% resumo em inglês
\begin{resumo}[Abstract]
%\begin{otherlanguage*}{english}
Resumo da dissertação em inglês.

\lipsum[4]

\vspace{\onelineskip}

\noindent
\textbf{Keywords:} \imprimirkeyword   
%\end{otherlanguage*}
\end{resumo}

% ---
% inserir lista de ilustrações
% ---
{\SingleSpacing
\pdfbookmark[0]{\listfigurename}{lof}
\listoffigures*
\cleardoublepage
}
% ---

% ---
% inserir lista de quadros
% ---
%\pdfbookmark[0]{\listofquadrosname}{loq}
%\listofquadros*
%\cleardoublepage
% ---

% ---
% inserir lista de tabelas
% ---
{\SingleSpacing
\pdfbookmark[0]{\listtablename}{lot}
\listoftables*
\cleardoublepage
}
% ---

% ---
% inserir lista de abreviaturas e siglas
% ---

\begin{siglas}
  \item[IPT] Instituto de Pesquisas Tecnológicas do Estado de São Paulo
\end{siglas}

% ---

% ---
% inserir lista de símbolos
% ---
%\begin{simbolos}
%  \item[$ \Gamma $] Letra grega Gama
%  \item[$ \Lambda $] Lambda
%  \item[$ \zeta $] Letra grega minúscula zeta
%  \item[$ \in $] Pertence
%\end{simbolos}
% ---

% ---
% inserir o sumario
% ---
{\SingleSpacing
\pdfbookmark[0]{\contentsname}{toc}
\tableofcontents*
\cleardoublepage
}
% ---

% ----------------------------------------------------------
% ELEMENTOS TEXTUAIS
% ----------------------------------------------------------
\textual
\OnehalfSpacing

\chapter{Capitulo 1} \label{capitulo1}
\lipsum[5]

\section{Seção 1}
\lipsum[6-10]
\chapter{Capitulo 2} \label{capitulo2}
\lipsum[11]
\section{Seção 1}
\lipsum[12-15]
\section{Seção 2}
\lipsum[16-20]

% ----------------------------------------------------------
% ELEMENTOS PÓS-TEXTUAIS
% ----------------------------------------------------------
\postextual

% ----------------------------------------------------------
% Referências bibliográficas
% ----------------------------------------------------------
\setboolean{isBib}{true} % Obriga o titulo das referências à esquerda

\setlength\bibitemsep{12pt} % Espaço de uma linha entre as referencias

\bibliography{Bibliografia,../../Mendeley/bib/Mestrado}

\setboolean{isBib}{false} % Desobriga o titulo das referências à esquerda

% ----------------------------------------------------------
% Apêndices
% ----------------------------------------------------------

\apendices
\partapendices
\chapter{Lorem ipsum dolor sit amet}

\lipsum[21-26]

% ----------------------------------------------------------
% Anexos
% ----------------------------------------------------------

\anexos
\partanexos

\chapter{Anexo 01}
\lipsum[26-32]

\end{document}